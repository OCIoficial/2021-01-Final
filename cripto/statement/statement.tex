\documentclass{oci}
\usepackage[utf8]{inputenc}
\usepackage{lipsum}
\usepackage{xcolor}

\title{El cifrado de César}
\newcommand{\todo}[1]{{\color{red}{#1}}}

\begin{document}
\begin{problemDescription}
César, ya cansado de que no lo incluyan en los bailes, ha decidido formar un club secreto
donde ahora él tendrá la oportunidad de excluir a todas las personas que lo han discriminado por
tantos años.
Para excluir completamente a quienes no pertenecen al club, César ha decido que toda comunicación
dentro del club debe ser encriptada.

César ha diseñado un sistema muy simple para encriptar sus mensajes.
En primer lugar, solo encriptará mensajes conformados por letras mayúsculas del alfabeto
inglés, es decir, los mensajes no contendrán tildes ni la letra Ñ.
Los mensajes pueden contener espacios, pero estos no deben ser modificados en el proceso de
encriptación.
Para encriptar el mensaje, César elige un número $k$ entre 0 y 25 que será utilizado como llave.
Posteriormente, cada letra del mensaje es reemplazada por la letra del abecedario que se
encuentra $k$ posiciones más adelante.
En caso de llegar al final del abecedario se debe continuar contando desde la $A$.
Específicamente, si enumeramos las letras del abecedario entre 0 y 25, la letra $i$ debe ser
reemplazada por la letra $(i+k)\% 26$.

Como ejemplo considera el siguiente mensaje
\begin{center}
  \texttt{SE HA DE ATREVER A ESCRIBIR CON PLUMA DE AVESTRUZ}
\end{center}
Para el $k=5$ el mensaje encriptado sería:
\begin{center}
  \texttt{XJ MF IJ FYWJAJW F JXHWNGNW HTS UQZRF IJ FAJXYWZE}
\end{center}

\end{problemDescription}

\section*{Tarea y puntaje}
Dado un mensaje encriptado con el método de César, tu tarea es determinar la llave que fue
utilizada para encriptarlo.
Tu solución será probada con distintos mensajes y obtendrás puntaje proporcional a la cantidad
de casos correctos siendo 100 el puntaje máximo.
Todos los mensajes corresponderán a fragmentos del libro Don Quijote de la Macha a los cuales se
les han eliminado los tildes y signos de puntuación.

Para resolver esta tarea puede serte útil la siguiente información de la frecuencia de aparición
\todo{fill with stats}

\begin{inputDescription}
    La entrada está descrita en dos líneas.
    La primera contiene un entero $N$ correspondiente a la cantidad de caracteres en el mensaje
    (incluyendo espacios).
    La segunda línea contiene una cadena de texto conteniendo solo espacios o letras mayúsculas
    del alfabeto inglés.
    La cadena de texto corresponderá siempre a un mensaje que fue encriptado utilizando el método
    descrito en el enunciado.

\end{inputDescription}

\begin{outputDescription}
    La salida debe contener un único entero entre 0 y 25 correspondiente a la llave que
    fue utilizada para encriptar el mensaje.
\end{outputDescription}


\begin{sampleDescription}
\sampleIO*[0.95]{sample-1}
\sampleIO*[0.95]{sample-2}
\end{sampleDescription}

\end{document}
