\documentclass{oci}
\usepackage[utf8]{inputenc}
\usepackage{lipsum}
\usepackage{tikz}
\usepackage{relsize}
\usepackage{calc}
\usepackage{xcolor}
\usetikzlibrary{shapes.geometric, fadings}

\newcommand{\todo}[1]{{\color{red} #1}}


\pgfdeclareradialshading{ring}{\pgfpoint{0cm}{0cm}}%
{rgb(0cm)=(0,0,0);
rgb(0.6cm)=(0,0,0);
rgb(0.75cm)=(1,1,1);
rgb(1cm)=(1, 1, 1)}

\begin{tikzfadingfrompicture}[name=ring]
  \fill[shading=ring] (0, 0) circle (1);
\end{tikzfadingfrompicture}

\tikzset{
  hex/.style={
    draw, shape=regular polygon,regular polygon sides=6,minimum size=26mm,
    inner sep=0pt,outer sep=0pt,shape border rotate=90, thick, font=\bf\relsize{4}
  },
}
\title{Catalonia}

\begin{document}
\begin{problemDescription}
Catalonia es una nación de comerciantes y para distribuir sus productos, los comerciantes tienen que
desplazarse entre ciudades constantemente.
Esto puede parecer una tarea sencilla, pero dada la peculiar distribución de las ciudades en
Catalonia puede resultar difícil desplazarse entre ellas.

La nación de Catalonia está conformada por infinitas ciudades de forma hexagonal.
Al centro de la nación se encuentra su capital y el resto de las ciudades están distribuidas
en \emph{anillos} al rededor de esta de forma que cada ciudad tiene 6 ciudades adyacentes.

Para hacer más fácil su navegación, los comerciantes de Catalonia han ideado un esquema
para enumerar las (infinitas) ciudades.
Partiendo desde 0, se enumera primero la capital, luego se enumeran las ciudades del primer anillo,
luego las del segundo y así formando una espiral infinita.
La siguiente figura muestra un ejemplo de la enumeración para los primeros tres anillos.

\newdimen\R
\R=1cm
\newcommand{\hex}[3][]{
  \begin{scope}[rotate=90, xshift=#3*1.5*\R, yshift=#2*-0.866*\R]
  \draw[thick,#1] (0:\R) \foreach \x in {60, 120,...,359} {
    --(\x:\R)
  } -- cycle(90:\R);
  \end{scope}
}
\newcommand{\pos}[2]{
  #1*0.866*\R, #2*1.5*\R
}
\begin{center}
\begin{tikzpicture}[scale=0.8, every node/.style={scale=0.8, font=\bf\relsize{3}}]
  \hex{ 0}{ 0}

  \hex{ 1}{ 1}
  \hex{-1}{ 1}
  \hex{-2}{ 0}
  \hex{-1}{-1}
  \hex{ 1}{-1}
  \hex{ 2}{ 0}

  \hex{ 3}{ 1}
  \hex{ 2}{ 2}
  \hex{ 0}{ 2}
  \hex{-2}{ 2}
  \hex{-3}{ 1}
  \hex{-4}{ 0}
  \hex{-3}{-1}
  \hex{-2}{-2}
  \hex{ 0}{-2}
  \hex{ 2}{-2}
  \hex{ 3}{-1}
  \hex{ 4}{ 0}

  \hex{ 5}{ 1}
  \hex{ 4}{ 2}
  \hex{ 3}{ 3}
  \hex{ 1}{ 3}
  \hex{-1}{ 3}
  \hex{-3}{ 3}
  \hex{-4}{ 2}
  \hex{-5}{ 1}
  \hex{-6}{ 0}
  \hex{-5}{-1}
  \hex{-4}{-2}
  \hex{-3}{-3}
  \hex{-1}{-3}
  \hex{ 1}{-3}
  \hex{ 3}{-3}
  \hex{ 4}{-2}
  \hex{ 5}{-1}
  \hex{ 6}{ 0}

  \node[text=purple] (N0)  at (\pos{ 0}{ 0}) {0};

  \node[text=blue] (N1)  at (\pos{ 1}{ 1}) {1};
  \node[text=blue] (N2)  at (\pos{-1}{ 1}) {2};
  \node[text=blue] (N3)  at (\pos{-2}{ 0}) {3};
  \node[text=blue] (N4)  at (\pos{-1}{-1}) {4};
  \node[text=blue] (N5)  at (\pos{ 1}{-1}) {5};
  \node[text=blue] (N6)  at (\pos{ 2}{ 0}) {6};

  \node[text=green] (N7)  at (\pos{ 3}{ 1}) {7};
  \node[text=green] (N8)  at (\pos{ 2}{ 2}) {8};
  \node[text=green] (N9)  at (\pos{ 0}{ 2}) {9};
  \node[text=green] (N10) at (\pos{-2}{ 2}) {10};
  \node[text=green] (N11) at (\pos{-3}{ 1}) {11};
  \node[text=green] (N12) at (\pos{-4}{ 0}) {12};
  \node[text=green] (N13) at (\pos{-3}{-1}) {13};
  \node[text=green] (N14) at (\pos{-2}{-2}) {14};
  \node[text=green] (N15) at (\pos{ 0}{-2}) {15};
  \node[text=green] (N16) at (\pos{ 2}{-2}) {16};
  \node[text=green] (N17) at (\pos{ 3}{-1}) {17};
  \node[text=green] (N18) at (\pos{ 4}{ 0}) {18};

  \node[text=orange] (N19) at (\pos{ 5}{ 1}) {19};
  \node[text=orange] (N20) at (\pos{ 4}{ 2}) {20};
  \node[text=orange] (N21) at (\pos{ 3}{ 3}) {21};
  \node[text=orange] (N22) at (\pos{ 1}{ 3}) {22};
  \node[text=orange] (N23) at (\pos{-1}{ 3}) {23};
  \node[text=orange] (N24) at (\pos{-3}{ 3}) {24};
  \node[text=orange] (N25) at (\pos{-4}{ 2}) {25};
  \node[text=orange] (N26) at (\pos{-5}{ 1}) {26};
  \node[text=orange] (N27) at (\pos{-6}{ 0}) {27};
  \node[text=orange] (N28) at (\pos{-5}{-1}) {28};
  \node[text=orange] (N29) at (\pos{-4}{-2}) {29};
  \node[text=orange] (N30) at (\pos{-3}{-3}) {30};
  \node[text=orange] (N31) at (\pos{-1}{-3}) {31};
  \node[text=orange] (N32) at (\pos{ 1}{-3}) {32};
  \node[text=orange] (N33) at (\pos{ 3}{-3}) {33};
  \node[text=orange] (N34) at (\pos{ 4}{-2}) {34};
  \node[text=orange] (N35) at (\pos{ 5}{-1}) {35};
  \node[text=orange] (N36) at (\pos{ 6}{ 0}) {36};

  % \draw[thick, ->] (N0) -- (N1);
  % \draw[thick, ->] (N1) -- (N2);
  % \draw[thick, ->] (N2) -- (N3);
  % \draw[thick, ->] (N3) -- (N4);
  % \draw[thick, ->] (N4) -- (N5);
  % \draw[thick, ->] (N5) -- (N6);
  % \draw[thick, ->] (N6) -- (N7);
  % \draw[thick, ->] (N7) -- (N8);
  % \draw[thick, ->] (N8) -- (N9);
  % \draw[thick, ->] (N9) -- (N10);
  % \draw[thick, ->] (N10) -- (N11);
  % \draw[thick, ->] (N11) -- (N12);
  % \draw[thick, ->] (N12) -- (N13);
  % \draw[thick, ->] (N13) -- (N14);
  % \draw[thick, ->] (N14) -- (N15);
  % \draw[thick, ->] (N15) -- (N16);
  % \draw[thick, ->] (N16) -- (N17);
  % \draw[thick, ->] (N17) -- (N18);
  % \draw[thick, ->] (N18) -- (N19);
  % \draw[thick, ->] (N19) -- (N20);
  % \draw[thick, ->] (N20) -- (N21);
  % \draw[thick, ->] (N21) -- (N22);
  % \draw[thick, ->] (N22) -- (N23);
  % \draw[thick, ->] (N23) -- (N24);
  % \draw[thick, ->] (N24) -- (N25);
  % \draw[thick, ->] (N25) -- (N26);
  % \draw[thick, ->] (N26) -- (N27);
  % \draw[thick, ->] (N27) -- (N28);
  % \draw[thick, ->] (N28) -- (N29);
  % \draw[thick, ->] (N29) -- (N30);
  % \draw[thick, ->] (N30) -- (N31);
  % \draw[thick, ->] (N31) -- (N32);
  % \draw[thick, ->] (N32) -- (N33);
  % \draw[thick, ->] (N33) -- (N34);
  % \draw[thick, ->] (N34) -- (N35);
  % \draw[thick, ->] (N35) -- (N36);
\end{tikzpicture}
\end{center}

Para desplazarse por Catalonia, un comerciante debe seguir un camino moviéndose
siempre entre ciudades adyacentes.
El movimiento entre dos ciudades adyacentes toma siempre una unidad de tiempo.
% Dadas dos ciudades, un comerciante siempre usará el camino que tome menos tiempo para
% desplazarse entre ellas.
Por ejemplo, para desplazarse entre las ciudades 3 y 8, un comerciante puede moverse primero
a la ciudad 0, luego a la 1 y finalmente a la 2 para un tiempo total igual a 3.

Dadas una ciudad inicial $a$ y una final $b$, tu tarea es determinar el tiempo mínimo
necesario para desplazarse entre $a$ y $b$.
\end{problemDescription}

\begin{inputDescription}
    La entrada contiene una línea con dos enteros $a$ y $b$ \todo{limites} correspondientes respectivamente
    a la ciudad inicial y final.
\end{inputDescription}

\begin{outputDescription}
    La salida debe contener un único entero correspondiente al tiempo mínimo necesario
    para desplazarse entre las ciudades $a$ y $b$. \todo{limite para respuesta}
\end{outputDescription}

\begin{scoreDescription}
  \subtask{10}
  Se probarán varios casos en que $a = 0$ y $b \leq 18$.
  \subtask{30}
  Se probarán varios casos en que el tiempo de desplazamiento entre $a$ y $b$ es menor que 100.
  \subtask{60}
  Se probarán varios casos sin restricciones adicionales.
\end{scoreDescription}

\begin{sampleDescription}
\sampleIO{sample-1}
\sampleIO{sample-2}
\end{sampleDescription}

\end{document}

\begin{center}
\begin{tikzpicture}[scale=0.7, every node/.style={scale=0.8}]
  \hex[fill=purple, opacity=0.6]{ 0}{ 0}

  \hex[fill=blue, opacity=0.6]{ 1}{ 1}
  \hex[fill=blue, opacity=0.6]{-1}{ 1}
  \hex[fill=blue, opacity=0.6]{-2}{ 0}
  \hex[fill=blue, opacity=0.6]{-1}{-1}
  \hex[fill=blue, opacity=0.6]{ 1}{-1}
  \hex[fill=blue, opacity=0.6]{ 2}{ 0}

  \hex[fill=green, opacity=0.6]{ 3}{ 1}
  \hex[fill=green, opacity=0.6]{ 2}{ 2}
  \hex[fill=green, opacity=0.6]{ 0}{ 2}
  \hex[fill=green, opacity=0.6]{-2}{ 2}
  \hex[fill=green, opacity=0.6]{-3}{ 1}
  \hex[fill=green, opacity=0.6]{-4}{ 0}
  \hex[fill=green, opacity=0.6]{-3}{-1}
  \hex[fill=green, opacity=0.6]{-2}{-2}
  \hex[fill=green, opacity=0.6]{ 0}{-2}
  \hex[fill=green, opacity=0.6]{ 2}{-2}
  \hex[fill=green, opacity=0.6]{ 3}{-1}
  \hex[fill=green, opacity=0.6]{ 4}{ 0}

  \hex[fill=orange, opacity=0.6]{ 5}{ 1}
  \hex[fill=orange, opacity=0.6]{ 4}{ 2}
  \hex[fill=orange, opacity=0.6]{ 3}{ 3}
  \hex[fill=orange, opacity=0.6]{ 1}{ 3}
  \hex[fill=orange, opacity=0.6]{-1}{ 3}
  \hex[fill=orange, opacity=0.6]{-3}{ 3}
  \hex[fill=orange, opacity=0.6]{-4}{ 2}
  \hex[fill=orange, opacity=0.6]{-5}{ 1}
  \hex[fill=orange, opacity=0.6]{-6}{ 0}
  \hex[fill=orange, opacity=0.6]{-5}{-1}
  \hex[fill=orange, opacity=0.6]{-4}{-2}
  \hex[fill=orange, opacity=0.6]{-3}{-3}
  \hex[fill=orange, opacity=0.6]{-1}{-3}
  \hex[fill=orange, opacity=0.6]{ 1}{-3}
  \hex[fill=orange, opacity=0.6]{ 3}{-3}
  \hex[fill=orange, opacity=0.6]{ 4}{-2}
  \hex[fill=orange, opacity=0.6]{ 5}{-1}
  \hex[fill=orange, opacity=0.6]{ 6}{ 0}

  % \node at (\pos{ 0}{ 0}) {0};
  % \node at (\pos{ 1}{ 1}) {1};
  % \node at (\pos{-1}{ 1}) {2};
  % \node at (\pos{-2}{ 0}) {3};
  % \node at (\pos{-1}{-1}) {4};
  % \node at (\pos{ 1}{-1}) {5};
  % \node at (\pos{ 2}{ 0}) {6};

  % \node at (\pos{ 3}{ 1}) {7};
  % \node at (\pos{ 2}{ 2}) {8};
  % \node at (\pos{ 0}{ 2}) {9};
  % \node at (\pos{-2}{ 2}) {10};
  % \node at (\pos{-3}{ 1}) {11};
  % \node at (\pos{-4}{ 0}) {12};
  % \node at (\pos{-3}{-1}) {13};
  % \node at (\pos{-2}{-2}) {14};
  % \node at (\pos{ 0}{-2}) {15};
  % \node at (\pos{ 2}{-2}) {16};
  % \node at (\pos{ 3}{-1}) {17};
  % \node at (\pos{ 4}{ 0}) {18};

  % \node at (\pos{ 5}{ 1}) {19};
  % \node at (\pos{ 4}{ 2}) {20};
  % \node at (\pos{ 3}{ 3}) {21};
  % \node at (\pos{ 1}{ 3}) {22};
  % \node at (\pos{-1}{ 3}) {23};
  % \node at (\pos{-3}{ 3}) {24};
  % \node at (\pos{-4}{ 2}) {25};
  % \node at (\pos{-5}{ 1}) {26};
  % \node at (\pos{-6}{ 0}) {27};
  % \node at (\pos{-4}{-2}) {29};
  % \node at (\pos{-5}{-1}) {28};
  % \node at (\pos{-3}{-3}) {30};
  % \node at (\pos{-1}{-3}) {31};
  % \node at (\pos{ 1}{-3}) {32};
  % \node at (\pos{ 3}{-3}) {33};
  % \node at (\pos{ 4}{-2}) {34};
  % \node at (\pos{ 5}{-1}) {35};
  % \node at (\pos{ 6}{ 0}) {36};
\end{tikzpicture}
\end{center}