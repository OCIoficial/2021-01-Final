\documentclass{oci}
\usepackage[utf8]{inputenc}
\usepackage{lipsum}
\usepackage{tikz}
\usepackage{relsize}
\usepackage{calc}
\usepackage{xcolor}
\usetikzlibrary{shapes.geometric, fadings}

\newcommand{\todo}[1]{{\color{red} #1}}

\definecolor{color1}{HTML}{6290ff}
\definecolor{color2}{HTML}{775ef0}
\definecolor{color3}{HTML}{dd2680}
\definecolor{color4}{HTML}{fe6100}
\definecolor{color5}{HTML}{ffb001}

\pgfdeclareradialshading{ring}{\pgfpoint{0cm}{0cm}}%
{rgb(0cm)=(0,0,0);
rgb(0.6cm)=(0,0,0);
rgb(0.75cm)=(1,1,1);
rgb(1cm)=(1, 1, 1)}

\begin{tikzfadingfrompicture}[name=ring]
  \fill[shading=ring] (0, 0) circle (1);
\end{tikzfadingfrompicture}

\tikzset{
  hex/.style={
    draw, shape=regular polygon,regular polygon sides=6,minimum size=26mm,
    inner sep=0pt,outer sep=0pt,shape border rotate=90, thick, font=\bf\relsize{1}
  },
}
\title{Catalonia}

\begin{document}
\begin{problemDescription}
Catalonia es una nación de comerciantes.
Para distribuir sus productos, los comerciantes deben constantemente
viajar entre ciudades.
Esto puede parecer una tarea sencilla, pero dada la peculiar, distribución de las ciudades
en Catalonia puede resultar difícil viajar entre ellas.

La nación de Catalonia está conformada por infinitas ciudades de forma hexagonal.
Al centro de la nación se encuentra su capital y el resto de las ciudades están distribuidas
en \emph{anillos} al rededor de esta de forma que cada ciudad tiene 6 ciudades adyacentes.

Para hacer más fácil su navegación, los comerciantes de Catalonia han ideado un esquema
para enumerar las (infinitas) ciudades.
Partiendo desde 0, se enumera primero la capital, luego se enumeran las ciudades del primer anillo,
luego las del segundo y así formando una espiral infinita.
La siguiente figura muestra la enumeración para los primeros tres anillos.

\newdimen\R
\R=1cm
\newcommand{\hex}[3][]{
  \begin{scope}[rotate=90, xshift=#3*1.5*\R, yshift=#2*-0.866*\R]
  \draw[thick,#1] (0:\R) \foreach \x in {60, 120,...,359} {
    --(\x:\R)
  } -- cycle(90:\R);
  \end{scope}
}
\newcommand{\pos}[2]{
  #1*0.866*\R, #2*1.5*\R
}
\begin{center}
\begin{tikzpicture}[scale=0.6, every node/.style={scale=0.6, font=\bf\relsize{3.5}}]
  \hex{ 0}{ 0}

  \hex{ 1}{ 1}
  \hex{-1}{ 1}
  \hex{-2}{ 0}
  \hex{-1}{-1}
  \hex{ 1}{-1}
  \hex{ 2}{ 0}

  \hex{ 3}{ 1}
  \hex{ 2}{ 2}
  \hex{ 0}{ 2}
  \hex{-2}{ 2}
  \hex{-3}{ 1}
  \hex{-4}{ 0}
  \hex{-3}{-1}
  \hex{-2}{-2}
  \hex{ 0}{-2}
  \hex{ 2}{-2}
  \hex{ 3}{-1}
  \hex{ 4}{ 0}

  \hex{ 5}{ 1}
  \hex{ 4}{ 2}
  \hex{ 3}{ 3}
  \hex{ 1}{ 3}
  \hex{-1}{ 3}
  \hex{-3}{ 3}
  \hex{-4}{ 2}
  \hex{-5}{ 1}
  \hex{-6}{ 0}
  \hex{-5}{-1}
  \hex{-4}{-2}
  \hex{-3}{-3}
  \hex{-1}{-3}
  \hex{ 1}{-3}
  \hex{ 3}{-3}
  \hex{ 4}{-2}
  \hex{ 5}{-1}
  \hex{ 6}{ 0}

  \node[text=color3] (N0)  at (\pos{ 0}{ 0}) {0};

  \node[text=color1] (N1)  at (\pos{ 1}{ 1}) {1};
  \node[text=color1] (N2)  at (\pos{-1}{ 1}) {2};
  \node[text=color1] (N3)  at (\pos{-2}{ 0}) {3};
  \node[text=color1] (N4)  at (\pos{-1}{-1}) {4};
  \node[text=color1] (N5)  at (\pos{ 1}{-1}) {5};
  \node[text=color1] (N6)  at (\pos{ 2}{ 0}) {6};

  \node[text=color5] (N7)  at (\pos{ 3}{ 1}) {7};
  \node[text=color5] (N8)  at (\pos{ 2}{ 2}) {8};
  \node[text=color5] (N9)  at (\pos{ 0}{ 2}) {9};
  \node[text=color5] (N10) at (\pos{-2}{ 2}) {10};
  \node[text=color5] (N11) at (\pos{-3}{ 1}) {11};
  \node[text=color5] (N12) at (\pos{-4}{ 0}) {12};
  \node[text=color5] (N13) at (\pos{-3}{-1}) {13};
  \node[text=color5] (N14) at (\pos{-2}{-2}) {14};
  \node[text=color5] (N15) at (\pos{ 0}{-2}) {15};
  \node[text=color5] (N16) at (\pos{ 2}{-2}) {16};
  \node[text=color5] (N17) at (\pos{ 3}{-1}) {17};
  \node[text=color5] (N18) at (\pos{ 4}{ 0}) {18};

  \node[text=color4] (N19) at (\pos{ 5}{ 1}) {19};
  \node[text=color4] (N20) at (\pos{ 4}{ 2}) {20};
  \node[text=color4] (N21) at (\pos{ 3}{ 3}) {21};
  \node[text=color4] (N22) at (\pos{ 1}{ 3}) {22};
  \node[text=color4] (N23) at (\pos{-1}{ 3}) {23};
  \node[text=color4] (N24) at (\pos{-3}{ 3}) {24};
  \node[text=color4] (N25) at (\pos{-4}{ 2}) {25};
  \node[text=color4] (N26) at (\pos{-5}{ 1}) {26};
  \node[text=color4] (N27) at (\pos{-6}{ 0}) {27};
  \node[text=color4] (N28) at (\pos{-5}{-1}) {28};
  \node[text=color4] (N29) at (\pos{-4}{-2}) {29};
  \node[text=color4] (N30) at (\pos{-3}{-3}) {30};
  \node[text=color4] (N31) at (\pos{-1}{-3}) {31};
  \node[text=color4] (N32) at (\pos{ 1}{-3}) {32};
  \node[text=color4] (N33) at (\pos{ 3}{-3}) {33};
  \node[text=color4] (N34) at (\pos{ 4}{-2}) {34};
  \node[text=color4] (N35) at (\pos{ 5}{-1}) {35};
  \node[text=color4] (N36) at (\pos{ 6}{ 0}) {36};
\end{tikzpicture}
\end{center}

Para viajar entre dos ciudades, un comerciante debe seguir un camino moviéndose
siempre entre ciudades adyacentes.
El movimiento entre dos ciudades adyacentes toma siempre una unidad de tiempo.
% Dadas dos ciudades, un comerciante siempre usará el camino que tome menos tiempo para
% desplazarse entre ellas.
Por ejemplo, para viajar desde la ciudad 3 a la 8, un comerciante puede moverse primero
a la ciudad 0, luego a la 1 y finalmente a la 2 para un tiempo total de viaje igual a 3.
De forma alternativa, puede también usarse el camino 3-2-1-8, en cuyo caso el tiempo total de viaje
es también 3.
El camino 3-11-2-1-8 también es posible, pero este tiene un tiempo total de viaje igual a 4.
En este caso el tiempo mínimo necesario para viajar entre 3 y 8 es 3, pues no hay ningún otro camino
más corto.

Dadas una ciudad inicial $a$ y una final $b$, tu tarea es determinar el tiempo mínimo
necesario para viajar desde $a$ hacia $b$.
\end{problemDescription}

\begin{inputDescription}
  La entrada consiste en una única línea con dos enteros $a$ y $b$ ($0\leq a < b \leq 10^{18}$)
  correspondientes respectivamente a la ciudad inicial y final.
\end{inputDescription}

\begin{outputDescription}
  La salida debe contener un único entero correspondiente al tiempo mínimo necesario
  para viajar desde la ciudad $a$ a la $b$.
\end{outputDescription}

\begin{scoreDescription}
  \subtask{a}
  Se probarán varios casos en que $a = 0$ y $b \leq 36$.
  \subtask{b}
  Se probarán varios casos en que $a = 0$ y $b \leq 10^18$.
  \subtask{c}
  Se probarán varios casos en que $a < b \leq 10^5$.
  \subtask{d}
  Se probarán varios casos sin restricciones adicionales.
\end{scoreDescription}

\begin{sampleDescription}
\sampleIO{sample-1}
\sampleIO{sample-2}
\end{sampleDescription}

\end{document}