\documentclass{oci}
\usepackage[utf8]{inputenc}
\usepackage{lipsum}
\usepackage{nicefrac}

\title{Torneo de ajedrez}

\begin{document}
\begin{problemDescription}
  Debido al éxito de la miniserie original de Netflix, Gambito de Dama, el ajedrez
  se está volviendo cada vez más popular.
  Decenas de nuevos torneos se organizan cada mes y la cantidad sigue en aumento.
  Dado este explosivo aumento, la asociación mundial de ajedrez se está comenzando a preocupar
  de la fiabilidad en los resultados de todos estos nuevos torneos.

  En un torneo de ajedrez cada participante debe enfrentarse a diferentes contrincantes y en cada
  partida recibirá puntaje dependiendo del resultado.
  Una partida puede terminar con uno de los dos contrincantes victorioso en cuyo caso el
  participante ganador recibe un punto (1) mientras que el perdedor recibe cero puntos (0).
  Adicionalmente, los contrincantes pueden decidir terminar en \emph{tablas} en cuyo caso ambos
  contrincantes reciben medio punto (\nicefrac{1}{2}).
  Dada estas reglas, es posible que al final de un torneo varios participantes terminen con
  la misma cantidad de puntos.
  Se considera como \emph{campeones} a todos los participantes que obtienen el puntaje máximo
  al final de un torneo.

  Dada una lista con los resultados de cada partida en un torneo, a la asociación
  mundial de ajedrez le gustaría determinar quienes son los campeones del torneo.
  ?`Podrías ayudarlos?
\end{problemDescription}

\begin{inputDescription}
  La primera línea de la entrada contiene dos enteros $N$ y $P$
  ($2 \leq N \leq 100, 0 < P \leq 1000$),
  correspondientes respectivamente a la cantidad de participantes y la cantidad de partidas
  en el torneo.
  Cada participante en el torneo es identificado con un entero entre 0 y $N-1$.

  Cada una de las siguientes $N$ líneas contiene tres enteros $A$, $B$ y $C$
  ($0\leq A < B < N$), describiendo el resultado de una partida.
  Los enteros $A$ y $B$ corresponden a los jugadores que se enfrentaron en esa partida.
  El entero $C$ indica el resultado de la partida.
  Si la partida terminó en tablas, $C$ será igual a {-1}.
  En caso contrario, $C$ será igual a $A$, indicando que $A$ fue el ganador de la partida, o igual
  a $B$, indicando que $B$ fue el ganador de la partida.
  Se garantiza que un jugador no se enfrentará más de una vez al mismo contrincante.
\end{inputDescription}

\begin{outputDescription}
  La salida debe contener la lista los campeones del torneo.
  Específicamente, si la cantidad de campeones es $W$, la salida debe contener
  $W$ líneas cada una con un entero correspondiente a uno de los campeones.
  La lista puede ser impresa en cualquier orden.
\end{outputDescription}

\begin{scoreDescription}
  Este problema no contiene subtareas.
  Se probarán varios casos y se entregará puntaje proporcional a la cantidad de casos correctos
  siendo 100 el puntaje máximo.
\end{scoreDescription}

\begin{sampleDescription}
\sampleIO[0.5][0.46]{sample-1}
\vspace{-1em}
En este ejemplo todos los participantes jugaron dos partidas y en todas estas empataron, es decir,
al final del torneo todos los jugadores terminaron con 1 punto y todos son campeones.
La salida debe por lo tanto contener a todos los jugadores (en cualquier orden).
\vspace{1em}

\sampleIO[0.5][0.46]{sample-2}
\vspace{-1em}
En este ejemplo el participante 0 ganó dos partidas para un total de 2 puntos, el 1 ganó una partida
para un total de 1 punto y el 2 no ganó ninguna partida para un total de 0 puntos.
Por lo tanto el participante 0 es el único campeón.
\end{sampleDescription}
\vspace{1em}

\end{document}
